% Section 3 in file constraints.tex

\todo{EXPAND: as per Richie's advice to use the PPT for section}

% Section 3.1
\subsection{Economic Factors}

% Section 3.2
\subsection{Time Limitations}

% Section 3.2.1
\subsubsection{Project Ramp-up}

% Section 3.2.2
\subsubsection{Summer Semester}

% Section 3.3
\subsection{Safety and Security}
\label{sec:safety-sec}
\todo{0.5 pages}

The safety and security of WHCS is primary constraint of the project. Due to
the integration with home, especially access control systems, WHCS must not
negatively affect home security. Additionally, as WHCS is in control of large
currents involving light control and outlet control, great care must be taken
to design circuits and software to prevent fires and misbehavior. If for
instance, we decided to control a light that exceeded the rating of one of the
control relays, then this could be a fire hazard. Also, in terms of door
control, if the mechanism for controlling the door were to fall in to the hands
of a burglar or fail completely this would present a critical safety and
security issue.

For example, there is a trade-off that needs to be made for controlling a door
with an electronic strike. Electronic strikes come in two main flavors:
normally opened (NO) and normally closed (NC). NO favors security by
\emph{failing-secure}, meaning the lock will not be openable if in the event of
a power loss. NC on the other hand will \emph{fail-safe}, meaning the lock will
be openable without power applied. This consideration should be based around
local fire code and depending on the type of door and handle used. We explain
our decision to go with a NO type electronic strike in
\autoref{sec:electronic-strike}.

In general, some principals for making sure that safety and security problems
are taken in to consideration are

\begin{enumerate} \item Analyze potential problem areas \item Develop solutions
for problem areas \item Anticipate failures and handle them accordingly
\end{enumerate}

Through premptive \emph{analysis}, we may determine problem areas. We will
\emph{develop} solutions for these problems, consisting of a description of the
problem, its potential impact, what area does it impact, and what we plan to do
to address it. Finally, we will \emph{anticipate} failures and build in the
developed solutions directly in to our design. These solutions may be in the
form of warning labels (Ex. DO NOT EXCEED 12V 10A) and software checks.

% Section 3.4
\subsection{Spectrum Considerations}
For WHCS we will need to provide over the air communication between the android
device to the base station and between the base station and the control
modules. Out of the frequency bands that the FCC has marked as unlicensed we
decided to use the band that includes 2.4 GHz. Although there are other
unlicensed bands that could have been used such as the 900MHz band and the 5GHz
band, 2.4GHz provides the best solution. The higher the frequency the shorter
the range yet the better the data rate and the smaller the device. The main
reason why 2.4GHz was chosen is because it is a happy median of good range and
acceptable data rates. Unfortunately because 2.4GHz is such a good frequency to
operate at it also has a lot of interference from other devices that operate at
this same frequency. However interference will happen from whatever frequency
band that is chosen, and this issue wasn{}'t enough of a problem for our design
to deter us from using the 2.4GHz band. Both the NRF chip and the Bluetooth
chip operate within this band.
