% Section 3 in file constraints.tex

% Section 3.1
\subsection{Economic Constraints}
Economic constraints are the biggest hindrance in the development of WHCS. The
amount of money necessary to complete such a project adds up quickly due to the
necessity of obtaining hardware like printed circuit boards. This project is
being developed by college students so the amount of reserve money that is
available is low. The average full{}-time student schedule can obstruct
students from having time to earn extra money. To help alleviate the lack of
funds our group applied for funding through Boeing. Boeing was kind enough to
approve our application for WHCS. Unfortunately we did not get all the money
that we asked for which means part of the project will have to be paid for out
of pocket.  This will especially be an issue if things go wrong during the
development process that cause us to have to repurchase certain pieces of
hardware. In all of our design time decisions cost plays the biggest factor.
Our research and development budget is small compared to mass{}-produced
products similar to WHCS. We have recognized this constraint from the offset
and we plan accordingly.

% Section 3.2
\subsection{Time Limitations}
The amount of time that we have to create WHCS is a limiting factor for what we
are trying to create. To begin with, the technologies used in this project are
not something that we were already well{}-versed in before starting. There was
a ramp{}-up period for figuring out what technologies would be necessary to
implement a system with the capabilities we desired. This is not even including
the research time necessary for finding exact chips that fit each of the
requirements we needed. During the development lifecycle of the project we are
not just creating WHCS we are learning things necessary to design in general
such as creating PCB layouts. Learning things like this, necessary to actually
create parts of the project take a measurable chunk of time to do. To add on
top of the ramp{}-up time that we have to endure, we will be missing valuable
weeks due to the summer semester. Our second semester of senior design will
take place in the summer semester which is 12 weeks instead of the normal 16
weeks of a semester. This means that we lose out on almost an entire month due
to the second half of development being in the summer. This whole month of lost
time will put a large strain on the group and it is something that we have
accounted for early on.

\subsection{Political Constraints}
There are no relevant political constraints that we have attributed to the
development of WHCS. The development of WHCS is not aligned with the ideals of
any political party. Our research results showed that there is not noticeable
political involvement in products similar to ours.

\subsection{Ethical, Environmental, and Sustainability Constraints}
The discussion of ethical constraints for WHCS is directly related to
environmental awareness and sustainability. The most ethically bearing
decisions we have had to make during the development of our system involve
power usage. There is no doubt that power usage has a vast effect on the
environment. While developing WHCS our goal when faced with multiple options
for implementation is to take the path which results in the least power
consumed. For example, the type of electronic strike we chose was based on
which one wasted less power during operation. We believe that making decisions
in this area to help the environment and promote sustainability give us the
best ethical outlook. This is also the only ethical involvement we have
attributed to our project.

\subsection{Manufacturability Constraints}

WHCS will have the potential to be implemented in many homes as long as
consideration is put into manufacturability during design time. As we make
decisions for our system we are constantly monitoring how it affects the
overall ease of replication. We have identified the base station PCB and
control module PCBs to be the biggest contributors of our overall
manufacturability. In order to maximize manufacturability we have constrained
ourselves to only using highly available parts for incorporation into our PCB
designs. Following through with this commitment will ensure that WHCS
installations will be standardized and easy to replicate.

% Section 3.3
\subsection{Safety and Security}
\label{sec:safety-sec}
\todo{0.5 pages}

The safety and security of WHCS is primary constraint of the project. Due to
the integration with home, especially access control systems, WHCS must not
negatively affect home security. Additionally, as WHCS is in control of large
currents involving light control and outlet control, great care must be taken
to design circuits and software to prevent fires and misbehavior. If for
instance, we decided to control a light that exceeded the rating of one of the
control relays, then this could be a fire hazard. Also, in terms of door
control, if the mechanism for controlling the door were to fall in to the hands
of a burglar or fail completely this would present a critical safety and
security issue.

For example, there is a trade-off that needs to be made for controlling a door
with an electronic strike. Electronic strikes come in two main flavors:
normally opened (NO) and normally closed (NC). NO favors security by
\emph{failing-secure}, meaning the lock will not be openable if in the event of
a power loss. NC on the other hand will \emph{fail-safe}, meaning the lock will
be openable without power applied. This consideration should be based around
local fire code and depending on the type of door and handle used. We explain
our decision to go with a NO type electronic strike in
\autoref{sec:electronic-strike}.

In general, some principals for making sure that safety and security problems
are taken in to consideration are

\begin{enumerate} \item Analyze potential problem areas \item Develop solutions
for problem areas \item Anticipate failures and handle them accordingly
\end{enumerate}

Through premptive \emph{analysis}, we may determine problem areas. We will
\emph{develop} solutions for these problems, consisting of a description of the
problem, its potential impact, what area does it impact, and what we plan to do
to address it. Finally, we will \emph{anticipate} failures and build in the
developed solutions directly in to our design. These solutions may be in the
form of warning labels (Ex. DO NOT EXCEED 12V 10A) and software checks.

% Section 3.4
\subsection{Spectrum Considerations}
For WHCS we will need to provide over the air communication between the android
device to the base station and between the base station and the control
modules. Out of the frequency bands that the FCC has marked as unlicensed we
decided to use the band that includes 2.4 GHz. Although there are other
unlicensed bands that could have been used such as the 900MHz band and the 5GHz
band, 2.4GHz provides the best solution. The higher the frequency the shorter
the range yet the better the data rate and the smaller the device. The main
reason why 2.4GHz was chosen is because it is a happy median of good range and
acceptable data rates. Unfortunately because 2.4GHz is such a good frequency to
operate at it also has a lot of interference from other devices that operate at
this same frequency. However interference will happen from whatever frequency
band that is chosen, and this issue wasn{}'t enough of a problem for our design
to deter us from using the 2.4GHz band. Both the NRF chip and the Bluetooth
chip operate within this band.
