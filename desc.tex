% Section 2 in file desc.tex

\subsection{Motivation}
The goal of this project is to improve the quality of life for people in their
homes. Imagine sitting on the couch at home about to watch a movie, but all the
lights are on and it{}'s a little warm inside. It is irksome to have to get up
and turn off every individual light. With the technology existing today it is
perfectly feasible to be able to turn off the lights and turn on a fan with a
mobile phone. With the software available today it is even possible for this
process to be initiated by voice. The problem that exists is these solutions
lack mass implementation. By creating a wireless home control base station that
a mobile phone could connect to these visions can be realized. The need to get
up and physically interact with an appliance can be made a thing of the past.

We want to develop an easy to use system that allows people at their home to
interact with their appliances without having to be in front of them. Our aim
is for the solution to be reliable and low cost. The use case scenarios should
be intuitive so that even someone who was just visiting could utilize the
system. A person using WHCS should be able to turn on their lights or an outlet
with the press of a button or with a voice command from their mobile phone.
They should also be able to turn on their coffee pot from their phone when they
first wake up. If someone knocks on the door the person should be able to
unlock the door without having to get up. With the activation capabilities of
WHCS there is an opportunity to utilize a foundation that can be expanded upon.
We will have the infrastructure for integrating different types of sensors into
the home to provide users with information about things like temperature or air
quality.

\subsection{Overview}
The diagram pictured in \autoref{fig:whcs-overview} shows the highest level
overview of WHCS.  When the user wants to begin interacting with WHCS he has
the option of choosing to use a mobile phone or the included LCD screen. Both
option will provide full capabilities for interacting with the system. The
phone will be attached to the system through a BlueTooth connection that is
created by the user in the WHCS application. Using the phone will be a more
mobile and easy method for access because the LCD will be connected to the
central component of WHCS, the base station. The base station will be the
brains of WHCS. It will be the base station{}'s job to take commands from the
user and relay them to the endpoints, while also displaying the state of the
system. The base station will have a list of endpoints, also called control
modules, that can be targeted by the system. This list will be dynamic and
allow for endpoints to be added or removed from the system during operation.
Together the base station and the control modules will form a network through a
home and will communicate wirelessly to one another through radio transceivers.

The control modules designed for WHCS will allow for all the activation of
appliances around the house. These endpoints will be listening for commands
from the base station via a radio transceiver. Each control module will be
tailored for interacting with a certain device. There will be control modules
for toggling outlets, toggling lights, unlocking the front door, and also for
monitoring sensors. The control modules will be as similar as possible with a
designated area that allows for assigning specific roles to the control
modules.

\ucfgfx{fig:whcs-overview}{a21expected2pages-img001.png}{WHCS System Overview}

% Section 2.2
\subsection{Objectives}
\todo{3 pages}
In order to enable homeowners to have the best experience with their new WHCS,
we will explain our core project objectives. These describe what the end-users
are be able to do with the system at a high level.

% Section 2.2.1
\subsubsection{Voice Control}
Voice control from a supported, BlueTooth enabled, Android device will allow
the user to remotely activate any part of the home that is integrated with
WHCS. This would include activating lights, unlocking doors, turning off and on
appliances (by controlling their respective outlets), querying sensors, and any
other home specific applications.\footnotemark All of these \emph{actions} and
\emph{targets} will be able to be used just from the user's voice. Voice
actions will be specific to each target, but they will consist of verbs such as
\textbf{turn}, \textbf{query}, \textbf{check}, \textbf{open}, \textbf{close},
and so on. The list of targets will directly correspond to the number of
control modules listed in the home and their type.  This will be explained in
more detail in \autoref{sec:sys-design}.

\footnotetext{WHCS is an extensible system. Control modules are built with a
plugin-like interface, allowing for intrepid home owners to have a fully custom
home. This combined with the control module's free breadboard area, new
applications may be created.}

% Section 2.2.2
\subsubsection{Light Activation}
Through activating lights and querying their status remotely, a homeowner will
no longer have to be present in the same room as the switch.  By connecting
lights to WHCS, they will become integrated in to the home network and not be
isolated in each room of the home.  With just a spoken command or a tap on
their smartphone, lights will be controlled.  By automating the process of
toggling light switches, WHCS will have the ability to be smart about when they
are ON or OFF, freeing the user from having to think about their state at all.
In addition, lights, just like all of the other connected control modules in
the home, will be able to be actuated on a specific schedule. This schedule
\todo{make a section for scheduling or remove} can be designed by the homeowner
to meet their daily lives or in a special circumstance, such as travel.

% Section 2.2.3
\subsubsection{Outlet Activation}
Lights aren't the only actionable thing in the home. There are a multitude of
appliances throughout the home which could benefit from remote control. Some
of these include coffee makers, toasters, or computers\todo{find some better
examples}. If integrated with WHCS through outlet control, these appliances
would be able to be apart of the home network. Imagine being able to start the
morning coffee from the comfort of the bedroom. This would be possible with an
appropriate coffee maker and WHCS outlet control. An added benefit from having
outlets being automated is that there would be less draw from power leeching
devices' power subsystems, which may be always-on. \todo{cite article
hyping up how much power you lose to leeches}

% Section 2.2.4
\subsubsection{Door Access}
In addition to controlling home lights and various appliances, giving users
remote control of their doors is  a goal of WHCS. Through the use of an
\emph{electronic door strike}, we would provide specific control module the
capability of locking and unlocking a door. This functionality would be
demonstrated in Demo \ref{sec:demo-door}. WHCS sees door access as important
for a home automation system to support because remote access, like a garage
door, is simple and easy. We want to make opening \emph{any} door simple and
easy.

Unlike controlling appliances and reading sensors, correctly managing the
operation of a safety-critical door must be handled with great care. Any flaw
in the implementation of the WHCS network would leave a user's home vulnerable
to outside attack. This is why any control module whose purpose is to control
doors will support additional security features. These additional features will
include mechanisms to prevent replay attacks (which garage doors are vulnerable
to) and also prevent outside attackers from engaging the door opening mechanism
just by sniffing traffic. For additional details on the security considerations
of WHCS, please refer to \autoref{sec:safety-sec}.

% Section 2.2.5
\subsubsection{Data Collection}
In order to give users a broad overview of their home's state, WHCS supports
the collection of data from \emph{arbitrary} sensors. Data collected can
include temperature, humidity, light level, sound levels, and so on. Each home
may have sensors throughout collecting various data that the homeowner deems
useful. The sensor integration with WHCS would be transparent to the user. All
they would see would be the list of sensors and the corresponding values.
WHCS's pluggable control module's would be tailored to each sensor or set of
sensors, which would relay their data back to the base station. This is
illustrated in \autoref{fig:many-to-one}. The base station would support
queries from the LCD interface and simultaneously from a connected phone.

\ucfgfx{fig:many-to-one}{many-to-one}{A illustration showing the translation of many different sensor nodes to WHCS' protocol}

Beyond home sensors, all of the other controllable objects in the WHCS
would have metadata being collected about them at all times. This extra metadata
would include \textbf{connection status} and \textbf{power status}. See
\autoref{sec:network-library} for a more detailed description of the supported
network packets and the type of fields they support.

All of this raw data being collected could be displayed to the user in the form
of graphs and tables. It would also serve as the basis for a set of descriptive
statistics for display to the user.

% Section 2.3
\subsection{Requirements and Specifications}

% Section 2.4
\subsection{Research of Related Products}

% Section 2.4.1
\subsubsection{Z-Wave}

% Section 2.4.2
\subsubsection{Belkin}

% Section 2.4.3
\subsubsection{Apple HomeKit}

% Section 2.4.4
\subsubsection{Nest Labs}
\todo{0.5 pages}
Unlike the previous companies, \href{http://nest.com}{Nest Labs} is quite new,
but has certainly claimed its space in the smart home market with its smart
Nest Thermostat. Their other product, the Nest Protect, a smart smoke and
CO$_2$ detector, integrates smoothly with the thermostat allowing for remote
monitoring and control of the home. For their thermostat, the primary goal is
to have a smart learning thermostat that aims to save energy in the home. By
keeping temperatures at energy-saving levels when no one through the use of a
motion sensor and learning algorithms, one of the most expensive home energy
costs can be reduced.

In terms of administration, Nest has a online web interface and a mobile app
that will display all of the networked devices, allowing for a highly connected
experience. Change your temperature from your warm bed or before you even get home!

Nest Labs was recently
\href{http://www.forbes.com/sites/greatspeculations/2014/01/17/googles-strategy-behind-the-3-2-billion-acquisition-of-nest-labs/}{purchased by Google for \$3.2 billion dollars},
which certainly gives the company a powerful position in the market. One of
Nest's goals is to have a \emph{platform} for other companies and developers to
create new products that \href{https://nest.com/works-with-nest/}{\emph{Work with Nest \texttrademark}}. This strategy is
clever as now the success of the company will grow with every new
developer who chooses to integrate their products with the Nest suite.
Customers will see the multitude of devices that work with Nest and realize
that they can ``harness the future today.'' Quite a solid business model.
Coming in at \$249, the thermostat might be a tough sell for typical home owners who
already have a working, yet ``dumb'', thermostat.

% Section 2.4.5
\subsubsection{X10}
\todo{0.5 pages}

