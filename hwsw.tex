% Section 5 in file hwsw.tex

% Section 5.1
\subsection{Radio Transceiver}

% Section 5.1.1
\subsubsection{Operating Principles and Usability}

% Section 5.1.2
\subsubsection{Driver Class Diagram}

% Section 5.1.3
\subsubsection{Driver Use Case}

% Section 5.1.4
\subsubsection{Network Library}
\label{sec:network-library}
\todo{3 pages}

% Section 5.2
\subsection{Microcontrollers}
\todo{Talk about pin count and having to disable the JTAG fuses to get access to those pins}
\todo{Talk about the speed grade for the CPU and why a higher grade would be faster}


% Section 5.2.1
\subsubsection{Development Environment}

% Section 5.2.2
\subsubsection{Programming}

% Section 5.2.3
\subsubsection{Control Module}

% Section 5.2.4
\subsubsection{Base Station}

% Section 5.2.5
\subsubsection{External Oscillator}

% Section 5.3
\subsection{BlueTooth Chip}

% Section 5.3.1
\subsubsection{RN-41}

% Section 5.3.2
\subsubsection{HC-05}

% Section 5.4
\subsection{LCD}
\todo{whole section}
Being able to interface with WHCS like a normal wall thermostat is one of our
project goals. Having a centralized display that can quickly display the most
important information for homeowners would be step up from traditional ``dumb''
thermostats. With a simple LCD combined with a touchscreen, users now have a
way to control and query their home without having to find their phone.

For our LCD touchscreen, we have choosen Adafruit's 2.8" TFT\footnotemark with
resistive touch, which uses the \href{}{ILI9341 chipset}. There are
\href{https://github.com/adafruit/Adafruit_ILI9341/tree/master/examples}{plenty
of usage examples} and Adafruit's
\href{https://learn.adafruit.com/adafruit-2-dot-8-color-tft-touchscreen-breakout-v2}{excellent
technical documentation} combined with their
\href{https://github.com/adafruit/Adafruit_ILI9341}{libraries} guarantee that
integrating this in to our design will be straight forward. One issue with this
solution is with the ILI9341 driver code: it was written to target the Arduino
platform. Now, the Arduino platform is fairly close to bare AVR, minus the
remapped pin numbers and some support libraries. Porting Adafruit's library
would be a feasible solution or writing a specific minimal driver would suffice
as well.

\footnotetext{\url{https://www.adafruit.com/products/1770}}

% Section 5.4.1
\subsubsection{Capabilities}
\todo{1 page}

% Section 5.4.2
\subsubsection{Driver}
\todo{2 page}

% Section 5.4.3
\subsubsection{Touchscreen}
\todo{1 page}

X+, X-, Y+, Y-

% Section 5.4.4
\subsubsection{UI Library}
\todo{2 page}

% Section 5.5
\subsection{Android Application}

% Section 5.5.1
\subsubsection{Development Environment}

% Section 5.5.2
\subsubsection{Speech Recognition}

% Section 5.5.3
\subsubsection{BlueTooth Library}

% Section 5.5.4
\subsubsection{GUI Philosophy}

% Section 5.5.5
\subsubsection{Use Case Diagram}

% Section 5.5.6
\subsubsection{BlueTooth Listener Class}

% Section 5.5.7
\subsubsection{Creating Endpoint Groups}

% Section 5.6
\subsection{Power Hardware}

% Section 5.6.1
\subsubsection{Design Summary}

% Section 5.6.2
\subsubsection{Regulators vs DC to DC Converters}

% Section 5.6.3
\subsubsection{Backup Battery Configuration}

% Section 5.6.4
\subsubsection{Transformer Choice}

% Section 5.6.5
\subsubsection{Power Consumption}

% Section 5.6.6
\subsubsection{Isolation}

% Section 5.6.7
\subsubsection{Power Through Hole Board}

% Section 5.6.8
\subsubsection{Schematic}

% Section 5.7
\subsection{Base Station}
\todo{whole section except sch breakdown}

% Section 5.7.1
\subsubsection{Software Flowchart}

% Section 5.7.2
\subsubsection{Control Module Data Structures}

% Section 5.7.3
\subsubsection{Networking State Machine}

% Section 5.7.4
\subsubsection{Associating With Base Station}

% Section 5.7.5
\subsubsection{Schematic Breakdown}

% Section 5.8
\subsection{Control Module}
\todo{whole section except schematic breakdown}

% Section 5.8.1
\subsubsection{Software Flowchart}

% Section 5.8.2
\subsubsection{Schematic Breakdown}

% Section 5.8.3
\subsubsection{High-Voltage Control}

% Section 5.8.4
\subsubsection{Electronic Strike}
\label{sec:electronic-strike}

% Section 5.8.4.1
\paragraph{Normally Open or Normally Closed}

% Section 5.8.4.2
\paragraph{Strike vs Deadbolt}

% Section 5.8.5
\subsubsection{Sensor Collection}

% Section 5.8.6
\subsubsection{Light Control}

