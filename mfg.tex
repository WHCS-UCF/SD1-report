\tbw

\subsection{PCB House}
\todo{1 page}

\subsubsection{OSH Park}

\subsubsection{Seeed Studio}

\subsubsection{4PCB}

\subsection{Parts}

\subsubsection{Footprint (SMD vs Through-Hole)}
\todo{this isn't complete from Joseph}
In WHCS we had to consider two construction methods for our base station and
control module boards; through hole boards, and surface mounted boards. Through
hole board technology is the older of the two technologies and is currently
much less popular than surface mounting. One of the of the advantages of
surface mounting is that it takes up less space allowing more real estate for
parts for a given board. Because surface mounting does not involve drilling it
is simpler and faster to construct. Although there are some advantages in
through hole boards for most applications surface mounted technology wins.
Therefore in our design we will be using surface mounted technology.

\subsection{Construction}

\subsubsection{Soldering}

\subsubsection{Reflow Oven}

\subsubsection{Proto-Panel}
\label{sec:proto-panel}
In this section we will focus on some of the details of the construction of our
design, the general overview of how the display board is to be put together is
given in \autoref{sec:proto-panel-proto} This section will not explain how the
proto{}-panel will come together, rather it will explain the non obvious
construction details.We will discuss the design considerations that have to be
accounted for. This section will take into consideration the norms that go into
home construction. This will go into the specifics of safety precautions as
well as regular sizes and spacing used in a home.

First let{}'s talk about the wiring. The amount of current that can be drawn
from an outlet is 15 amps if not 20 amps, therefore we will have no problem in
drawing enough current from the outlet to power our board. 15 amps is more than
enough to satisfy our needs even if the current is divided into five different
applications. Other things to consider is the gauge of the wire used. The gauge
of the wire is dependent on the amount of current it can safely handle. As
discussed previously 1 amp is larger than anything we expect to see from our
circuit. Therefore if we design our wire gauge for 1 amp we should be fine.
However as stated before most homes are designed to be able to draw 15 if not
20 amps. This level of amperable is equivalent with wire of gauge 14 and 12. A
common brand of wiring used for these tyes of applications is Romex. Just for
the sake of being consistent with what is used in the home we{}'ll use 12 or 14
gauge wire. To splice the wire we could either solder it or we could use a wire
nut. Both are an acceptable method for joining the wires, and both are used in
homes. Wire nuts are considered to be an easier/ faster method for doing the
job. While soldering and using a heat sink is seen as the higher performing
link. We will make our link by soldering because it is slightly more
professional than using a wire nut but really this is simply a matter of
preference.

When wiring something it is often a good idea to attach your wire at more
locations than simply the location where the connection is made. This way if
for whatever reason the wire is pulled the stress will not go completely to the
connection. Before splicing the wire and connecting the wires to the different
control modules and base station, it would be a smart safety precaution to run
the unspliced wire through the wood framing and attach it. After splitting the
wire it would also be a smart idea to continue the practice of attaching the
wire at more places than the connections. Although not completely necessary it
would be nice if the color of our wires followed the understood meanings of
wire colors in home wiring. In our design we have a single phase hot and
neutral and ground. In the US the ground wire matches with green, the black
wire matches with the hot wire, and the white wire matches with neutral.
\footnote{\url{http://www.allaboutcircuits.com/vol_5/chpt_2/2.html}}
Using these color codes will make it easier to keep the project neat and
organized. It will make it easier to avoid mistakes made in wiring a circuit

Now let{}'s discuss the boards themselves and how they will be placed into the
framework of our wall. The boards will require some sort of casing that can
hopefully easily be attached to the framework of the home. The less the number
of things needed for installation the better. The easiest way to do this will
be to attach them to or place them inside of the electrical boxes that we are
using for our applications. Having only one thing to install per control module
or base station will make real installation of our device more realistic for
actual use in the future. In our design we will make use of electrical boxes
since they are used in home electrical wiring. The boxes can be either metal or
plastic, yet plastic boxes are a little easier to work with as they the holes
are easier to punch out.
\footnote{\url{http://homerenovations.about.com/od/electrical/a/artelecbox.htm}}

There are some specific considerations that must be made with the different
control modules. First off for the light it is important that we use an actual
wall mounted light, as this is the type of light fixture that we are hoping to
be controlling in an actual home. It is important that it is not simply a plug,
because this would defeat the purpose of having light fixture control module.
The wall mounted light should come with three wires; a hot, a neutral, and a
ground wire. From these three wires we will be able to install the fixture in
the same way as what would be expected in a real home. The hot wire is the wire
that we will be interrupting with the relay in order to switch the light on and
off.

The outlet we will be using is a GFIC. GFIC stands for ground fault
interrupter. Using this outlet will provide extra safety precaution. What a
GFIC outlet does is constantly compare the output current from the neutral wire
to the input current from the hot wire. If there{}'s a difference in current,
within the range of a few milliamps, the outlet will shut off in 20-30
milliseconds. In the case that someone were to be electrocuted by this outlet,
the current that goes into their body would cause a current leakage that would
cause the GFIC to have a current difference and thus shut off.  GFIC outlet are
normally required for kitchen and bathrooms. Since none of the members in our
group have extensive experience in working with AC power it is best that we
take every safety precaution available. We do not expect that homes that
actually implement our design to use GFIC outlets, it is simply an extra safety
precaution that our group decided to take. \footnote{\url{http://diy.stackexchange.com/questions/15684/what-is-a-gfi-outlet-used-for-and-where-should-i-install-them}}


For the door control module we decided to make our own door of the size 18'{}'
by 18.'{}' Since we are not buying the door but are custom making it ourselves
we need to cut the holes ourselves. The first order of business is cutting the
door to length and leaving a frame of the right size. To cut the hole of the
latch we will need to use a $\frac78$'' spade bit. The hole for the door knob
will have to be made with 2-$\frac18$'' diameter hole saw. A 1'' wide chisel is
used to cut out the recess of the latch. We are now able to install the door
knob.
