% Section 6 in file pcb.tex

% Section 6.1
\subsection{Software Considerations}
\todo{1 more page if possible}
Before designing any of our Printed Circuit Boards, we decided to analyze which
software would allow us to do the job the quickest and easiest. Nearly all of
the team was familiar with EAGLE as it's one of the most talked about board
design software due to its EAGLE Lite version. Instead of going with the most
common solution, we decided to compare EAGLE CAD to another open source
solution: KiCad.

% Section 6.1.2
\subsubsection{EAGLE}
EAGLE PCB is commercial software for schematic capture and board layout. It
supports a wide variety of features that would help us make our board. The only
issue is that the normal software costs money. Luckily, they offer a free
evaluation version that can only be used for non-commercial purposes.

This freeware version of EAGLE has strict limitations in the size of the board
that can be designed and how many signal layers there may be. The size of any
board is limited to 4 x 3.2
inches\footnote{\url{http://www.cadsoftusa.com/download-eagle/freeware/}} and
there may only be a top and bottom copper layer. These limitations would be a
show stopper for a moderately complex board, but considering our project
requirements, it would be suitable.  If we are to consider future board designs
for WHCS, we may want a more flexible solution.

% Section 6.1.1
\subsubsection{KiCad}
As an alternative to EAGLE PCB, KiCad performs admirably well. It has all of
the primary features of EAGLE and yet, is completely free and open source. The benefit of this is that the whole suite of tools is cross platform, allowing group members to easily work together despite different operating systems.

One issue with KiCad is the lack of a built in Autorouter. KiCad provides an external router, FreeRouting\footnote{\url{http://www.freerouting.net/}}, but it has experienced recent legal trouble due to one of the developers previous employers.

Another nifty feature that KiCad has is its 3D board view. This feature is great for getting a sense of your board layout in relation to the selected footprints.

% Section 6.2
\subsection{Layout}

% Section 6.2.1
\subsubsection{Base Station}

% Section 6.2.2
\subsubsection{Control Modules}

