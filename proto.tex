\subsection{Point-To-Point Transmission}
The most essential part of WHCS is the ability for the modules and the base
station to be able to communicate wirelessly. In our research and prototyping
phase we made sure that this feat would be achievable. To ensure that we were
able to communicate using our radio transceivers we set up a prototype for
point{}-to{}-point transmission. The setup involved the use of two breadboards
each populated with a microcontroller and a radio transceiver. One
microcontroller out of the two was operating as the symbolic base station. The
HC{}-05 BlueTooth module was connected to the Atmega328 microcontroller and
through this module we were able to administer tests with the prototype setup.
The microcontroller acting as the base station had a routine that enabled
reading and writing to the NRF24L01's registers.  We were able to ensure that
the state of the radio transceiver is the state that we needed to communicate.
The other microcontroller was connected to the other radio transceiver as well
as a TTL{}-serial module for connecting to a computer{}'s terminal. The control
module microcontroller also had LED{}'s connected to two of the GPIO pins. The
setup that we created is shown in \autoref{fig:ptp-tx} This setup allowed for
us to make sure that the radio transceivers were able to send packets to one
another and that the packets could be read into the microcontrollers.

\ucfgfx[width=15.24cm,height=8.572cm]{fig:ptp-tx}{a81PointtoPointTransmissionexpected1pg-img001.png}{Point to Point Transmission Prototyping Setup}

When we finished connecting and hooking up this prototype setup
we were able to run the routine on the microcontroller that acted as the base
station to take input from the BlueTooth module. Using an Android phone we
communicated with the microcontroller and manipulated registers of the NRF24L01
to ensure that the microcontroller was communicating with the device using the
correct timing protocol. Once we ensured we were correctly interfacing with the
NRF24L01 from the base stations side we connected the NRF24L01 to the control
module chip. The control module chip was connected to a computer terminal via
the TTL{}-serial chip located in the top center of Fig. 8.1.1. We made sure
that we were able to interface with the radio transceiver for this
microcontroller in the same way as the first. Once both radio transceivers were
confirmed to be connected and interacting correctly we ordered the base station
to transmit data to the control module.  The command was initiating from the
Android application. The data was transmitted to the control module and was
successfully received. As a result, one of the LEDs attached to the control
module breadboard toggled to an on state.

This prototype was able to give us a good gauge of how feasible
our approach to designing a wireless home automation solution was. We
successfully able to get an Android device to communicate to a stripped down
base station and then to a stripped down control module. The actions that were
done with this prototype will be the core of WHCS. Every action done in the
system centers around the ability to communicate between microcontrollers and
to and from the mobile phone.  When the control module microcontroller is doing
more than just toggling an LED is when WHCS will be impressive.

\subsection{Rogers Board Etching Prototyping}

\subsection{WHCS Proto-Panel}
\label{sec:proto-panel-proto}
Presentation plays a huge part on how the public feels about a product. This is
why it is so important that the display of the project is well put together.
Not only is it important to have a nice display for the purpose of it being a
proper representation of our design, but also so that it is aesthetically
appealing. If WHCS were to be launched into industry, marketing would play a
huge part in it{}'s success. People{}'s first impressions are always driven by
what they see. If what they see causes them to believe that the product is of
high quality, they are less likely to be highly skeptical of how the product
performs. In this section we will be into details about what plans will be made
in order to showcase the functions of WHCS. This section will not go into the
practical side of how the display is coming together; rather it will lay the framework for what goals we want to
accomplish with the display.

Our design is not a plug and play design, therefore we won{}'t
be able to simply plug in the system. We{}'ll have to actually find some way to
duplicate what the installation of a house would look like. In a home what
we{}'d be provided with is interior wiring. When we are actually presenting our
idea what we{}'ll be presented with is an outlet that we can use to draw power
from. Therefore the first step we{}'d need to take is to convert this outlet
back into wiring. We could do this by simply using a basic cord. This cord will
provide us with a hot and a neutral wire and a path to ground. These wires will
then be used to power the control modules and the base station. We need to
somehow splice this wire into multiple wires.

We want to make the display as accurate of a representation of
what would be found in a home as possible. Therefore we will try to follow as
many codes and standards for home construction as possible. Since we ourselves
do not have a homeowners electrical permit we are not equipped to actually
install the system in a real home. Yet for demonstrative purposes we{}'ll do
fine to follow codes and present them in our display. In \autoref{sec:proto-panel}
we{}'ll go into further detail of what was done to follow these codes and
standards.

Our display will consist of the frame, the wiring, and drywall.
Each made to follow standards. Now our design will consist of 5 different
wirings; four control modules and one base station.

\subsubsection{Materials}
The first thing we need is a plug. It is important that we use a three prong
plug because wiring in the home uses three wires. In addition to the wiring the
interior walls of homes consist of drywall, insulation, and a wooden frame. The
wooden frame of homes is made out of 2 by 4 wood. These pieces of 2 by 4 wood
are usually put together with either screws or nails. Drywall will also be
needed to provide the presentation side of our wall. For drywall all we{}'ll
need is enough drywall to cover the entire wall along with some drywall screws
in order to attach the drywall. Insulation will be unnecessary since we
aren{}'t worried about temperature or sound insulation for our project. Also
insulation in a regular home won{}'t interfere with the installation of our
project so really the insulation is irrelevant.

In addition to these basic materials made to construct the walls
we{}'ll need a few other things. First we{}'ll need the actual control module
and base station boards. these boards along with the power board will be housed
in a case that will be attached to the wooden framework. For the light control
module we will need a wall mounted lamp with a three wire connection. For the
outlet we will obviously need to buy an outlet. For the outlet and light
control modules the relays will be placed along the hot wires in order to
switch them on and back off. To display the outlet control module we will need
to have something plugged into our display board outlet (for example a fan or a
coffee maker). The other three modules will need the LCD screen, the door knob,
the strike, the temperature sensor.


\subsubsection{Dimensions} The dimension of our project is pretty arbitrary,
that is as long as it is large enough to fix each control module along with the
base station. The first thing that really had to be decided was whether or not
we wanted to use a full size door for our design. Although the idea was
tempting, because it would really give the user the feel of a home experience,
we decided against it mostly because of weight. If the fame had to be of that
size all the wood used in the frame along with the weight of an actual door
would make the project very difficult to move around. Also we{}'d have to
consider sheer bulkiness of it. Getting an entire door (actually even larger
than an entire door because of the other attached components) through a door
can be quite a struggle, add weight to the mess and you{}'re asking for
difficulties.

The door we will be using will be a homemade door. Fortunately
because we are also designing the frame that will be used for the wall, we can
simply make the gap in between the studs the same size as the door we wish to
make. We decided that 18 inches by 18 inches is a suitable size to make the
door. After deciding this we figured that 2 feet by 2 feet would be a more than
large enough area of space to display each module. The total square footage of
the five control modules would be 20 square feet. We decided that a 4 by 5
display wall would showcase our design quite nicely.

\subsubsection{Sketch} In the 4 by 5 display we have to decide where everything
will go. The most important interactive parts of our design are the access
control module and the base station. We must make sure that these two modules
are at an acceptable height where they can be interacted with. We decided to
place three of the modules on the bottom portion and two modules on the top
portion. The two on top at a more accessible height therefore the door access
and the base station will be placed here. The light outlet and temperature
sensor will be placed on the bottom half. To make things look symmetrical the
lamp will be placed in the middle while the outlet and the temperature sensor
will be placed on the side. The display will look something like \autoref{fig:proto-sketch}.

\ucfgfx[width=15.24cm,height=9.878cm]{fig:proto-sketch}{a83expected2pagesprotopanel-img001.png}{General design of WHCS display board}
