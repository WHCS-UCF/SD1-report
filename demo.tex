\subsection{Voice Controlled Light Activation}
The first demo that we would like to perform with the functioning prototype of
WHCS will be voice controlled light activation. Voice control is one of the big
features of WHCS because it adds a lot of interest to any project. Light
activation through the system will be one of the most common use cases. Voice
controlled light activation combines an exciting feature with one of the most
common use cases so it will be a frequent demo. To perform this demo the
Android application will have to be paired with the base station already and
ready to communicate to the system. We can allow any person that wishes to
participate to utilize the application and access the voice command feature.
The person performing the voice activation will be able to say something
similar to on or off and toggle the state of a light in the system. This demo
can be extended to include outlets as well. We will have a coffee pot hooked up
to an outlet being controlled by WHCS. After activating the light the performer
will be able to turn off the light and then start interacting with the outlet.
The outlet can be turned on, thus turning on the coffee pot. This is a
preferable way to start the day so it should be a relatable demo for an
audience. From this point we can show that this feature is also accessible
through the GUI of the Android application. The voice chat feature is only a
replication of the GUI capabilities and this is an important point to make.

An extension that would help this demo would be letting participants create
their own commands for interacting with WHCS. This is a feature that we will be
incorporating into the Android application. This demonstrates how WHCS is
designed to be customizable for each household. A participant of the team will
be able to add a setting that merely saying mouse turns off all the lights and
outlets connected to the system. Then the performer would say mouse into the
voice control system and the result should be the lights and outlets of the
display being turned off. The point of this demo is to show the ease of use and
the power of the voice control in the Android application.

\subsection{LCD Light Activation}
\todo{1 page}

\subsection{Sensor Query}
Sensors are a passive part of WHCS that enhance the overall appeal of the
system but do not demo as well as turning things on or off. Our prototype of
WHCS will have a functioning temperature sensor control module that is updating
based on the temperature of the environment. In our demos we want to make sure
to show the operation of this sensor.  WHCS allows sensors like temperature
sensors to be mobile around the home. The sensor can be plugged in anywhere and
receive power and update for the temperature of that environment. When we are
demoing WHCS we want to show this mobility by showing the sensor being moved
around and updating accordingly. The demo will show how the sensors information
is viewable from the LCD as well as the Android application. If the sensors
data is not recent enough we can prompt the sensor to update its most recent
reading and this should be performed during the demo. This demo could be made
more powerful for viewers if the system reacted to changes in the sensors
reading. At this point in time we have not considered methods for WHCS to react
to sensor data. For our purposes the sensors will only serve as information
providers for users.

\subsection{Fault Recovery (Loss of Power)}
\todo{1 page}

\subsection{Remote Door Access}
\label{sec:demo-door}
The electric strike of WHCS will provide us with a door access demo. This will
be a great demo because it is very appealing to be able to unlock your door
wirelessly and it will show off the scalability of our system. The electric
strike will be physically mounted to the proto{}-panel that we will be using
with our demo. While the command to unlock the strike has not been given it
will be impossible to open the imitation door that is attached to the electric
strike.  The point of this demo will be to show that from the Android
application it will be possible to activate the electric strike and therefore
allow the door to be opened. The electric strike will also be available to be
unlocked from the LCD. During this demo we will explain that the electric
strike that we use for WHCS does not consume power while it is in the locked
state. It is only drawing current when it is toggled to the open state and this
happens for a maximum of thirty seconds. This will also be a great time to
mention how the utilization of an electric strike does not impede normal home
access. As long as the doorknob on the door has a key lock on it it can remain
in the locked position at all times. if the electric strike ever fails and is
unaccessible from the application then the door knob can simply be unlocked and
bypass the electric strike.
